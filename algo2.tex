\documentclass[12pt]{article}
\usepackage[utf8]{inputenc}
\usepackage[english,russian]{babel}
\usepackage{graphicx}
\usepackage{amsmath}
\usepackage{amsthm}
\usepackage{caption}
\usepackage{tikz}
\usepackage{subcaption}
\usepackage{imakeidx}
\usepackage[russian]{cleveref}
\usepackage[a4paper,left=15mm,right=15mm,top=30mm,bottom=20mm]{geometry}
\parindent=0mm
\parskip=3mm

\makeindex
\pagestyle{empty}

\title{ДЗ 2, Алгоритмы}
\author{Ушакова Алина}
\date{14.09.2019}

\begin{document}

\maketitle

\begin{enumerate}
\item % Задание 1
a)$2^k \leq n - 1$ => $k = [\log_2(n - 1)]$

$2^0 + 2^1 + 2^2 + ... + 2^k = \frac{1 \cdot (1 - 2^{k + 1})}{1 - 2} = 2^{k + 1} - 1 = 2 \cdot 2^{[\log_2 (n - 1)] + 1} - 1 = 4\cdot 2^{[\log_2 (n - 1)]} - 1 \sim  4\cdot n - 3 = \Theta(n)$

b) $\log 3 + ... + \log(\left[ \frac{n - 1}{3} \right] \cdot 3) = \log (3 \cdot ... \cdot (\left[\frac{n - 1}{3} \right] \cdot 3)) = \log(3^{\left[\frac{n - 1}{3}\right]} \cdot (1\cdot ... \cdot \left[\frac{n - 1}{3}\right])) = 
\log(3^{\left[\frac{n - 1}{3}\right]} \cdot \left[\frac{n - 1}{3}\right]!)$

$\sim \Theta(n) + \Theta(n!) = \Theta(n) + \Theta(n\log n) = \Theta(n \log n)$

c) Т.к. округление целочисленное, то внутренний цикл начинает выполняться при $i > [\sqrt{n}]$. Просуммируем кол-во действий, выполняемых во внутреннем цикле: 
\[\sum\limits_{i = \sqrt{n}}^{n}\left(i - \frac{n}{i}\right) = n\cdot\sum\limits_{i = \sqrt{n}}^{n}\left( \frac{i}{n} - \frac{1}{i}\right) = n\cdot\sum\limits_{i = \sqrt{n}}^{n}\frac{i}{n} - n\cdot\sum\limits_{i = \sqrt{n}}^{n}\frac{1}{i} = 
\sum\limits_{i = \sqrt{n}}^{n} i - n\cdot\sum\limits_{i = \sqrt{n}}^{n}\frac{1}{i} = \Theta(n^2) - o(n^2) = \Theta(n^2)\]

d)$\Theta(n)$, т.к. обе переменные(x и y) проходят от 1 до n по 1 разу.
\item % Задание 2
a) \%MOD работает долго, но мы можем постараться использовать его реже, делая проверку:

long long t = a[i] * b[i]:

if (t >= MOD)

    \qquad t \%= MOD;
    
sum += t;

if (sum >= MOD)

\qquad sum -= MOD;

b) Поскольку каждый раз считать pow долго, то можно создать массив степеней x(который мы посчитаем заранее pow\_x[i] = pow\_x[i - 1] * x), а потом не  считать заново, а просто обращаться к нужному значению. Рекурсию стоит заменить на цикл, т.к. это позволит нам использовать меньше памяти и совершать меньше действий(тогда можно даже не хванить массив pow\_, т.к. предыдущую степень можно вычислить)

double Exp(double x, int dep=0, double F=1) 

\{

\qquad if (dep >= 20) return 0;

\qquad double pow\_[20];

\qquad pow\_[0] = 1;

\qquad for (int i = 1; i < 20; i++)

\qquad\qquad pow\_[i] = pow\_[i - 1] * x;

\qquad double ans = 0;

\qquad for (int i = dep; i < 20; i++)

\qquad\{

\qquad\qquad ans += pow\_[dep] / F;

\qquad\qquad F *= dep + 1;

\qquad\qquad dep += 1;

\qquad\}

\qquad return ans;
%return pow(x, dep) / F + Exp(x, dep+1, F*(dep+1));

\}


c) Сложение строк работает долго(за длину строки), а в данной задаче это не имеет смысла => будет лучше выводить результать сразу. 

\item % Задание 3
Заметим, что при добавлении нового элемента к отрезку, сумма может только увеличиться(т.к. все элементы положительны). Будем поддерживать 2 конца отрезка(l - левый и r - правый) и сумму sum на отрезке. Если сумма меньше той, которую нам надо получить(s), то мы добавляем новый элемент к отрезку(добавляем к sum значение нового элемента и сдвигаем r). Если же sum > s, то нам нужно убрать элемент из начала отрезка.

   int n, s;

    cin >> n >> s;
    
    vector<int> v(n + 1);
    
    for (int i = 1; i <= n; i++)
    
        \qquad cin >> v[i];
        
    int sum = v[1];
    
    int l = 1, r = 1;
    
    while (r <= n \&\& l <= n)
    
    \{
    
        \qquad if (sum == s)
        
        \qquad \{
            \qquad cout << l << ' ' << r;
            
            \qquad \qquad return 0;
            
        \qquad \}
        
        \qquad if (sum < s)
        
        \qquad \{
        
           \qquad \qquad  r++;
            
            \qquad \qquad sum += v[r];
            
        \qquad \}
        
        \qquad else
        
        \qquad \{
        
           \qquad \qquad  sum -= v[l];
            
            \qquad \qquad l++;
            
        \qquad \}
        
    \}

%Дополнительная часть:
\item % Задание 1 из доп
a) $\int\limits_{k = 1}^{n}\frac{1}{k^2}dk = \left.-\frac{1}{k}\right| _{1}^{n} = 1 - \frac{1}{n} = \mathcal{O}(1)$

b)$\int\limits_{k = 1}^{n}\frac{1}{k^{1/2}}dk = \left.2k^{\frac{1}{2}}\right|_1^{n} = 2\sqrt{n} - 2 = \Theta(\sqrt{n})$

c)$\int\limits_{k = 1}^{\infty}\frac{1}{k^{3/2}}dk = \left.\frac{-2}{\sqrt{k}}\right|_1^{\infty} = 2 = \mathcal{O}(1)$
\item % Задание 2 из доп
\item % Задание 3 из доп
\item % Задание 4 из доп

\end{enumerate}

\end{document}